\documentclass{article}
\usepackage{ifthen,graphicx}
\usepackage[top=0.6in, bottom=0.9in, left=0.8in, right=0.8in]{geometry}
%\usepackage[top=1.0in, bottom=1.0in]{geometry}

%\pagestyle{empty}
\newboolean{KEY}
%\setboolean{KEY}{true}   %prints questions and answers
\setboolean{KEY}{false} %prints questions only
\newcommand{\answer}[1]{\ifthenelse{\boolean{KEY}}{#1}{}}
\newcommand{\titleheader}[2]{\ifthenelse{\boolean{KEY}}{#1}{#2}}

\begin{document}
\titleheader{\section*{CS170 SP2019 Quiz 3 Solution}}{\section*{CS170 SP2019 Quiz 3}}

This is a close book, close note quiz. Total points are 10. You have 15 minutes. Don't forget to put your name on the quiz. 

\begin{enumerate}

\item Briefly answer the following questions: [5]
\begin{enumerate}
%  \item What is a shell? What is a shell script.
%  \vspace{0.5in}
%  \answer{\emph{Shell is a command interpreter which will take user's command and communicate with the kernel to execute. Shell script is a simple program for shell to run.}}
  
%  \item Argument is a token that a command acts on. There are zero argument command, one argument command and two arguments command. List at least 3 for each type of command. 
%  \begin{enumerate} 
%  \item Zero argument: 
%  \answer{\emph{pwd, ls, hostname}} 
%  \item One argument: 
%  \answer{\emph{cat, cd, echo}}
%  \item Two arguments: 
%  \answer{\emph{cp, mv, grep, ls}}
%  \end{enumerate}
%	\vspace{0.1in}
\item The shell script can work with command-line arguments. Explain the following symbols appear in the shell script.
\begin{enumerate}
\item $\$0$
\item $\$1$
\item $\$2$
\item $\$*$
\item $\$\#$
\end{enumerate}
\vspace{0.2in}

  \item What are the two major modes of vim editor?
  \vspace{0.5in}
  \answer{\emph{Command mode and input mode.}}
  
	\item What is a pipe?
	\vspace{0.5in}
	\answer{\emph{Consists of one or more commands separated by a pipe symbol(|). }}
	
	\item What is a filter?
	\vspace{0.5in}
	\answer{\emph{A command that processes an input stream of data to produce an output stream of data.}}
  \end{enumerate}

\item What do the following Unix commands do? [5]  
\begin{enumerate}
\item cat part1 part2 part3 $>$ book
\vspace{0.3in}
\item who $>>$ whoson
\vspace{0.3in}
\item command1 $<$ someFile 
\vspace{0.3in}
\item ./program2
\vspace{0.3in}
\item cat file1 $|$ sort $|$ less
\vspace{0.3in}  
\item ps $|$ grep bash
\vspace{0.2in}
\item ls -l $|$ lpr $\&$
\vspace{0.3in}
\item kill 7324
\vspace{0.3in}
\end{enumerate}

\answer{\begin{enumerate}
\item \emph{A book file has been created to contain the content of three files: part1, part2 and part3}

\item \emph{Appending to file whoson with online user list}

\item \emph{Using someFile as the input for command1}

\item \emph{Executing program2}

\item \emph{show sorted the file1 page-wise}

\item \emph{search bash in the current process list}

\item \emph{Stop a process with ID 7324}
\end{enumerate}}

%\item Assuming the following files are in the working directory: 
%
%\begin{tabular}{l l l l l l}
%$\$ \texttt{ls}$ & & & & &\\
%intro & notesb & ref2 & section1 & section2 & section4b \\
%notesa & ref1  & ref3 & section2 & section4a & sentrev \\
%\end{tabular}

%What are the results after executing the following commands? [6]
%\begin{enumerate}
%\item \texttt{ls sec*}
%\vspace{0.5in}
%\item \texttt{ls ref?} 
%\vspace{0.5in}
%\item \texttt{ls section[1-3]}  
%\vspace{0.5in}
%\item \texttt{ls i*}
%\vspace{0.5in}
%\item \texttt{ls *[13]} 
%\vspace{0.5in}
%\item \texttt{ls note?[ab]} 
%\vspace{0.5in}
%\end{enumerate}
%
%
%\answer{\begin{enumerate}
%\item \emph{section1 section3 section4b section2 section4a}
%
%\item \emph{ref2 ref1 ref3}
%
%\item \emph{section1 section2 section3}
%
%\item \emph{intro}
%
%\item \emph{section1 ref1 ref3}
%
%\item \emph{notesa notesb}
%
%\end{enumerate}}

\end{enumerate}
\end{document}